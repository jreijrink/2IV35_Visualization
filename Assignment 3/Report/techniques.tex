\section{Techniques}\label{sec:techniques}


% Consider various interactive visualization techniques and combinations thereof
% You should have at least three different techniques in your approach that are linked interactively and are shown simultaneously.
% refer to design principles from slides!!!!


% map: useful for specific questions "Do restaurants receive a better rating when they are located in a certain area?" and "are there consumers that travel a long distance to visit a certain restaurant? And how do they travel and confirm they rate the restaurant well"

In order to achieve the tasks that were defined in the previous section, we will argue how suitable different visualization techniques are for doing so. We start with considering the first of these two tasks that is formulated as "Are there any correlations between restaurant ratings and one or more of their attribute values?". For this task, fairly simple visualization techniques could suffice, because we simply want to explore the relation between two attribute values (i.e., the restaurant its rating and one of its attribute values). Examples of suitable visualization techniques for this task are a Scatterplot \cite{D3scatterplot} or (Normalized) Bar Chart. More involved techniques such as a Parallel Coordinate\cite{} Plot could also be considered, these can show more than two attributes. Often these techniques have a tradeoff between the amount of data they are able to show, and the level of detail of the presented data. Because the data for the first is rather small, a more basic technique might even be preferable.

When looking at the second analysis task however, the amount of data we consider increases quite a lot. We especially want to be able to view more data simultaneously, because we want to link restaurant attributes to consumer attributes, using the rating that the consumer gave the restaurant. This means that we want at least three data objects to be shown: the restaurant ID, the consumer ID, and the rating. However, we could still do this using a basic Scatterplot by placing the ID's on the axes and introducing color coding for the rating. If we would want to explore more attribute relations at once, we could even consider to use a Scatterplot Matrix \cite{D3scattermatrix}. Though this increases the amount of space that is needed to show all of the plots and makes it harder to interpret them, this issue could be anticipated on by introducing an additional view that shows a single plot in detail.

Another technique that could be used to explore attribute relations is a Parallel Coordinate Plot \cite{D3pcp}. With this technique we could use three axes, where the left axis represents a restaurant attribute, the right axis a consumer attribute, and the middle one the rating that the consumer gave to that restaurant. In other words, each line would represent a rating by a consumer with a certain attribute value on a restaurant with a certain attribute value. However, because most of the attribute values are discrete, Parallel Sets \cite{D3parallelsets} might be considered as a better option. This technique on the other hand disables the analyst to easily distinguish single ratings.

<<<<<<< HEAD
Let us now consider the specific questions "Do restaurants receive a better rating when they are located in a certain area?" and "Are there consumers that travel a long distance to visit a certain restaurant? And how do they travel and confirm they rate the restaurant well". These questions clearly refer to a geospatial analysis. Therefore some kind of geographic (symbol) map \cite{D3map1} might be suitable in these cases in order to quickly get a clear overview of the data. But also in other cases a map might be interesting. For the restaurant \& consumer data set we could for instance use Mike Bostock his Symbol Map to get a quick overview of where all consumer live that rated a selected restaurant. Additionally we could show the rating of that consumer using a tooltip when the analyst hovers over the line that connects the restaurant to the consumer.
=======
Let us now consider the specific questions "Do restaurants receive a better rating when they are located in a certain area?" and "Are there consumers that travel a long distance to visit a certain restaurant? And how do they travel and confirm they rate the restaurant well". These questions clearly refer to a geospatial analysis. Therefore some kind of geographic (symbol) map \cite{D3map1} might be suitable in these cases in order to quickly get a clear overview of the data. But also in other cases a map might be interesting. For the restaurant \& consumer data set we could for instance use Mike Bostock his Symbol Map to get a quick overview of where all consumer live that rated a selected restaurant. Additionally we could show the rating of that consumer using a tooltip when the analyst hovers over the line that connects the restaurant to the consumer.




>>>>>>> 15d3dfd7c033d331ea0306bbd49900b07823930b
