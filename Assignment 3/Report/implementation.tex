\section{Implementation}\label{sec:implemenation}

For the implementation we decided to create a multiview application, consisting of 4 views.
The first view shows a parallel coordinate plot.
We decided to use this technique because we want to show more than 2 attributes.
We want to show a user selected consumer and restaurant attribute plus the rating of that consumer for the specific restaurant.
We could also use colors for the rating, but most attributes in the data set only have a few possible values, so there is a lot of overlap.
The result of this overlap is that not all colors are visible.
A parallel coordinate plot also shows dense regions better than other discussed techniques.
The parallel coordinate plot allows the user to select a region on the axis to filter data.

The second view is a map with all restaurants, users and their relation.
When a filter is applied ..

This interactive visualization will be able to answers both tasks to the user.


%\subsection{Techniques Task 1}\label{sec:techniques1}

%\subsection{Techniques Task 2}\label{sec:techniques2}

