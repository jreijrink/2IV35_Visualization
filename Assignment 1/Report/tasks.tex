\section{Tasks}

In this section we define the tasks that we would like to perform. That is, we describe what information we would like to retrieve from the given data set. Before we define these tasks we provide a framework (or design space) for formulating specific tasks.

\subsection{Framework for Defining Tasks}
In order to formulate a nice and clear visualization task, we define a framework for doing this as described in \cite{schulz2013design}. Here we distinguish between five different dimensions for each task, namely the \textit{goal} (why is the task pursued), \textit{means} (how is the task carried out), \textit{characteristics}(what does the task seek), \textit{target} (on which part of the data is the task carried out) and \textit{cardinality} (on how many instances is the task carried out). We shortly explain these five aspects in some more detail.

For the goal of the task we distinguish between three different types of analyses. We can have an exploratory analysis (or undirected search) that aims at deriving an hypothesis from an unknown data set, a confirmatory analysis ()

\todo{!}...

\subsection{Chosen Tasks}
As we now have defined a common framework for formulating visualization tasks, we move on to defining the actual tasks we want to perform on our own data set about the Dutch municipalities. We mainly distinguish between two different tasks (though the first one has a rather broad character).

\subsubsection{Taks 1}
The first task(s) that we define aims at discovering interesting properties of the given data set, as well as confirming expected properties. This means that this task is actually rather broad, and numerous specific questions can be formulated from it. However, we will define two basic (sub)tasks for analyzing the entire data set, namely a)'analyze the relations between the different attributes' in the set, and b) 'compare the attribute values of the different municipalities with each other'. We will define the same subset of attributes on which to perform for both of these tasks. These will be the attributes OAD, STED, AANT_INW, BEV_DICHTH, AANTAL_HH, P_EENP_HH, and OPP_TOT. For convenience we will call this set of attributes $S_{attr}$.

When formally defining task 1a, we observe that we adopt an \textit{exploratory} approach, as we aim at \textit{searching} interesting \textit{relations} between different data set attributes. From this we can (partly) derive the first three dimensions for our 5-tuple task description. The forth dimensions, the target has also been specified already, namely $s_{attr}$. This leaves us with the cardinality. Because we want to discover general relations, we want to look at all data instances (municipalities). However, some relations between attributes may be very obvious. We will therefore also be looking for \textit{confirmation} of these obvious relations, and want to detect possible abnormalities or inconsistencies in them, such as \textit{outliers}. The resulting formal task description then looks like this:

(exploratory|confirmatory, relation-seeking, relations|outliers, $S_{attr}$, all)

For task 1b we can define a somewhat similar task. The main difference is now that we will not be looking at the relations between the different data attributes, but are \textit{comparing} the data values of the different data objects. In this way we again hope to find interesting characteristics such as \textit{outliers} or \textit{clusters}. As with task 1a, we again take $S_{attr}$ as our target and all instances as cardinality.

(exploratory|confirmatory, comparison, outliers|clusters, $S_{attr}$, all)

Specific questions that we could pose for the first of these two tasks ...
\todo{specific questions}







\subsubsection{Taks 2}
In this task we search for confirmation to a specific question.
This question is formulated based on a hypothesis on which we seek validation.
Once the question is answered it can be used to support the hypotheses.\\
\\
The specific question we want an answer to is whether there are municipalities in the Netherlands that suffer from a high degree of an aging population ('vergrijzing' in Dutch).
This question is based on the hypothesis that the Netherlands is suffering of a high degree aging population.\\
The degree of aging population is based on the inhabitants that are older than 65 years, and the working inhabitants.
The working inhabitants are defined as the age group from 20 till 65, this is not the actual group of working inhabitants but it is a good representation.
\\
The specific question can be described as a formal task description:\\
(\textit{confirmatory, searching localization, attr($P\_00\_14\_JR$, $P\_15\_19\_JR$, $P\_20\_24\_JR$, $P\_25\_44\_JR$, $P\_45\_64\_JR$, $P\_65\_EO\_JR$), all})\\
\\
The formal task describes that the user is searching confirmation among all available 65+-percentage attribute values and the 20- till 64-percentage attribute values, this last attribute is created with a combination of different percentage attributes.
The task is confirmatory, so the user knows what he is looking for and seeking confirmation to what he wants to prove.
The user is looking at low-level data characteristics (localization), this means no complex patterns in the data but simply searching for the high valued attribute values in the data objects (municipalities).