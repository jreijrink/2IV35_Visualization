\section{Tasks}\label{sec:tasks}

In this section we define the visualization tasks that we would like to perform on the given data set. That is, we describe what information we would like to retrieve from the given data. However, before we define these tasks, we provide a framework (i.e., design space) for formulating visualization tasks.

\subsection{Framework for Defining Tasks}
In order to formulate a nice and clear visualization task, we provide a framework for doing so as presented in \cite{schulz2013design}. Here we distinguish between five different dimensions for each task, namely the \textit{goal} (why is the task pursued?), \textit{means} (how is the task carried out?), \textit{characteristics} (what does the task seek?), \textit{target} (on which part of the data is the task carried out?) and \textit{cardinality} (on how many instances is the task carried out?). We shortly explain these five aspects in some more detail.

For the goal of the task we distinguish between three different types of analyses. We can have an exploratory analysis (or undirected search) that aims at deriving an hypothesis from an unknown data set, a confirmatory analysis (directed search) that aims to verify a found or assumed hypothesis, and finally a presentation that simple exhibits analysis results.

For describing how the task is being carried out we distinguish between three main ways of doing so. These are navigation, (re-)organization and relation. Navigation subsumes all means that change the extent or the granularity of the shown data, but that do not reorganize the data itself. (Re-)organization includes all means that actually adjust the data to be shown by either reducing or enriching it, and relation encompasses all means that put data in context.

Characteristics of the data can be subdivided in low-level data characteristics and high-level ones. Examples of low-level characteristics are data values of a particular data object or the data objects corresponding to a particular data value. Examples of high-level characteristics include more complex data patterns such as trends, outliers, clusters, frequency, distribution, correlation, etc.

The target of the visualization task describes on which part of the data the task is being carried out. This often boils down to a certain (sub)set of the different data attributes.

The cardinality finally specifies which part of the data instances we consider when carrying out the task. For instance just a single one, a certain subset, or just all of them.


\subsection{Chosen Tasks}
With this common framework for formulating visualization tasks defined, we move on to defining the actual tasks we want to perform on our data set about the Dutch municipalities. We mainly distinguish between two different tasks, of which the first one will be divided into two subtasks.

\subsubsection{Task 1}\label{sec:task1}
The first task that aims at discovering interesting properties of the given data set, as well as confirming expected properties. This means that this task is rather broad, and numerous specific questions can be formulated from it. However, we will define two basic (sub)tasks for this task of analyzing the entire data set. These two subtasks are defined as \textit{a)'analyze the relations between the different attributes in the set'} , and \textit{b) 'compare the attribute values of the different municipalities with each other'}. The subset of attributes on which we will perform both of these tasks will be the same. This subset will consist of the attributes \texttt{OAD}, \texttt{STED}, \texttt{AANT\_INW}, \texttt{BEV\_DICHTH}, \texttt{AANTAL\_HH}, \texttt{P\_EENP\_HH}, and \texttt{OPP\_TOT}. For convenience we will call this set of attributes $S_{attr}$.

When formally defining task 1a, we observe that we adopt an \textit{exploratory} approach, as we aim at \textit{searching} interesting \textit{relations} between different data attributes. From this we can (partly) derive the first three dimensions for our 5-tuple task description. The forth dimensions, the target, has also been specified already, namely by $S_{attr}$. This leaves us with the cardinality. Because we want to discover general relations, we want to look at all data instances (municipalities). However, some relations between attributes may be very obvious. We will therefore also be looking for \textit{confirmation} of these obvious relations, and want to detect possible abnormalities or inconsistencies in them, such as \textit{outliers}. The resulting formal task description then looks like this:

(exploratory$|$confirmatory, relation-seeking, relations$|$outliers, $S_{attr}$, all)

For task 1b we can define a somewhat similar task. The main difference is now that we will not be looking at the relations between the different data attributes, but are \textit{comparing} the data values of the different data objects. In this way we again hope to find interesting characteristics such as \textit{outliers} or \textit{clusters}. As with task 1a, we take $S_{attr}$ as our target and all instances as cardinality.

(exploratory$|$confirmatory, comparison, outliers$|$clusters, $S_{attr}$, all)

Specific questions that we could pose for the first of these two subtasks is for instance 'is there a positive relation between the attributes \texttt{AANT\_INW} and \texttt{BEV\_DICHTH}?'. If we think about this for a minute in advance, we may predict that this will indeed be the case. This would hence be a typical question where we try to confirm our hypothesis, and look for data objects that do not adhere to this. A much more general and exploratorive question would be 'which attributes have a typical positive or negative relation with each other?'. Again we may define some expected relations, such as a negative relation between \texttt{OAD} and \texttt{STED}. However, this question really aims at an elaborate (though high level) analysis of the data.

For task 1b we can also define a question of which we may already predict the answer. We could for instance try to confirm that the four largest cities in The Netherlands score highest on the \texttt{AANT\_INW} attribute. We may even expect these values to be rather significantly larger than those of the other municipalities. Finally we can again pose a general, explorative question in the form of 'how are the attribute values of the municipalities distributed for a certain attribute, and can we explain this distribution?'.


\subsubsection{Task 2}\label{sec:task2}
The second task we define concerns the topic of a high degree of aging population ('\textit{vergrijzing}' in Dutch). The task is somewhat twofold, because we want to confirm that there are some municipalities in The Netherlands that suffer from a high degree of aging, as well as find out which municipalities these are. The degree of aging is defined by the percentage of inhabitants that are older than 65 years, compared to the percentage of working inhabitants. Because we do not have the exact percentage of working inhabitants of each population, we assume here that this percentage is somewhat similar to that of the percentage inhabitants aged between 20 and 65 years.

When formally defining this task we observe that we both want confirm an hypothesis as well as explore the data (explained above). Hence we will be searching for, and localizing the municipalities that suffer the most from a high degree of aging. The set of attributes that we will consider for this task consist of \texttt{P\_00\_14\_JR}, \texttt{P\_15\_19\_JR}, \texttt{P\_20\_24\_JR}, \texttt{P\_25\_44\_JR}, \texttt{P\_45\_64\_JR} and \texttt{P\_65\_EO\_JR}. This set will from here on be denoted by $T_{attr}$. Similar as for task 1, we again want to consider all of the municipalities for this task. Together this results in the following formal definition of task 2:

(confirmatory$|$exploratory, searching$|$localization, $T_{attr}$, all)

Though the specific question for this task was already mentioned earlier ('Which municipalities in The Netherlands suffer from a high degree of aging?'), we may also ask ourselves 'Is there a certain trend in the municipalities that suffer from a high degree of aging?'.
