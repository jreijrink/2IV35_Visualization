\section{Tasks}

In this section we define the tasks that we would like to perform. That is, we describe what information we would like to retrieve from the given data set. Before we define these tasks we provide a framework (or design space) for formulating specific tasks.

\subsection{Framework for Defining Tasks}
In order to formulate a nice and clear visualization task, we define a framework for doing this as described in \cite{schulz2013design}. Here we distinguish between five different dimensions for each task, namely the \textit{goal} (why is the task pursued), \textit{means} (how is the task carried out), \textit{characteristics}(what does the task seek), \textit{target} (on which part of the data is the task carried out) and \textit{cardinality} (on how many instances is the task carried out). We shortly explain these five aspects in some more detail.

For the goal of the task we distinguish between three different types of analyses. We can have an exploratory analysis (or undirected search) that aims at deriving an hypothesis from an unknown data set, a confirmatory analysis ()

\todo{!}...

\subsection{Chosen Tasks}
As we now have defined a common framework for formulating visualization tasks, we move on to defining the actual tasks we want to perform on our own data set about the Dutch municipalities. We mainly distinguish between two different task (though the first one has a rather broad character).

\subsubsection{Taks 1}
The first task(s) that we define aims at discovering interesting properties of the given data set, as well as confirming expected properties. This means that this task is actually rather broad, and numerous specific questions can be formulated from it. 

analyze (discover/confirm) relations between attributes
    -> detect outliers
    -> detect data inconsistencies
    -> detect trends
(exploratory|confirmatory, search?, relations|trends|outliers, attr(), all)

compare attribute values of different municipalities
    -> 
(exploratory|confirmatory, 

\todo{specific questions}







\subsubsection{Taks 2}
In this task we search for confirmation to a specific question.
This question is formulated based on a hypothesis on which we seek validation.
Once the question is answered it can be used to support the hypotheses.\\
\\
The specific question we want an answer to is whether there are municipalities in the Netherlands that suffer from a high degree of an aging population ('vergrijzing' in Dutch).
This question is based on the hypotheses that the Netherlands is suffering from a high degree of an aging population.\\
\\
The specific question can be described as a formal tuple:\\
(\textit{confirmatory, searching localization, attr($P\_65\_EO$), all})\\
\\
The formal task describes that the user is searching confirmation among all available 65+-percentage attribute values.
The task is confirmatory, so the user knows what he is looking for and seeking confirmation to what he wants to prove.
The user is looking at low-level data characteristics (localization), this means no complex patterns in the data but simply searching for the high valued attribute values in the data objects (municipalities).
